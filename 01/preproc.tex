\section{Препроцессор C/C++}

\subsection{PreProcessor / предобработчик кода}

\begin{frame}[t]{PreProcessor / предобработчик кода}

  Цель --- подготовка C/C++ кода к компиляции:
  
  \begin{itemize}
    \item замена соответствующих диграфов и триграфов на эквивалентные символы <<\#>> и <<\textbackslash>>.
    \item удаление экранированных символов перевода строки;
    \item замена строчных и блочных комментариев пустыми строками (с удалением окружающих пробелов и символов табуляции);
    \item вставка (включение) содержимого произвольного файла (\texttt{\#include});
    \item макроподстановки (\texttt{\#define});
    \item условная компиляция (\texttt{\#if}, \texttt{\#ifdef}, \texttt{\#elif}, \texttt{\#else}, \texttt{\#endif});
    \item вывод сообщений (\texttt{\#warning}, \texttt{\#error}).
  \end{itemize}
\end{frame}

\subsection{Ключевые слова препроцессора}

\begin{frame}[t]{Ключевые слова препроцессора}

  \begin{itemize}
    \item \texttt{\#define} --- создание константы или макроса;
    \item \texttt{\#undef} --- удаление константы или макроса;
    \item \texttt{\#include} --- вставка содержимого указанного файла;
    \item \texttt{\#if} --- проверка истинности выражения;
    \item \texttt{\#ifdef} --- проверка существования константы или макроса;
    \item \texttt{\#ifndef} --- проверка не существования константы или макроса;
    \item \texttt{\#else} --- ветка условной компиляции при ложности выражения \texttt{\#if};
    \item \texttt{\#elif} --- проверка истинности другого выражения; краткая форма записи для комбинации \texttt{\#else} и \texttt{\#if};
    \item \texttt{\#endif} --- конец ветки условной компиляции;
    \item \texttt{\#line} --- указание имени файла и номера текущей строки для компилятора;
    \item \texttt{\#error} --- вывод сообщения и остановка компиляции;
    \item \texttt{\#warning} --- вывод сообщения без остановки компиляции;
    \item \texttt{\#pragma} --- указание действия, зависящего от реализации, для препроцессора или компилятора;
  \end{itemize}
\end{frame}

\begin{frame}[t]{Ключевые слова препроцессора 2}

Если ключевое слово не указано, директива игнорируется;
Если указано несуществующее ключевое слово, выводится сообщение об ошибке и компиляция прерывается.



\end{frame}

\begin{frame}[t,fragile]{Ключевые слова препроцессора 2}

Если ключевое слово не указано, директива игнорируется;
Если указано несуществующее ключевое слово, выводится сообщение об ошибке и компиляция прерывается.

\end{frame}



