\section{<<Hello world!>> на C и на C++. Отличия С и C++}

\subsection{Чистый C}

\begin{frame}[t]{Первая программа на чистом C}
  \lstinputlisting[language=C]{01_first_C/main.c}
\end{frame}

\subsection{С++}

\begin{frame}[t]{Ввод данных на C++}
  \lstinputlisting[language=C++]{00_first/main.cpp}
\end{frame}

\begin{frame}[t,fragile]{Общая структура программы}

\begin{lstlisting}
// Подключение библиотек
#include <stdio.h>
// Все необходимые препроцессорные директивы

int main() { // начало главной функции с именем main
  // здесь будет находится ваш программный код
}
\end{lstlisting}

\end{frame}

\begin{frame}[t,fragile]{Подключение библиотек}
\begin{lstlisting}
#include <iostream>
\end{lstlisting}
директива препроцессора, предназначена для включения в
 исходный текст содержимое заголовочного файла, имя которого<iostream>,
 содержащий описания функций стандартной библиотеки ввода/вывода для работы
с клавиатурой и экраном.

По простому: Без этой строчки не будут работать функции для вывода текста на экран
И ввода с клавиатуры. Писать обязательно во всех программах.
\end{frame}


\begin{frame}[t,fragile]{Подключение библиотек}
\begin{lstlisting}
using namespace std; 
\end{lstlisting}
 директива означает что, все определённые ниже имена будут
относится к пространству имён std.

В частности, объекты для ввода данных и вывода на консоль 
находятся в этом пространстве имён.

\begin{lstlisting}
int a;
std::cin >> a;
std::cout << "a = " << a;
\end{lstlisting}

\end{frame}

\begin{frame}[t,fragile]{Основная функция программы}

Называется \texttt{main} и должна возвращать \texttt{int}

void означает что функция не возвращает никаких значений

\begin{lstlisting}
int main()  
{	         
  // Здесь находится собственно программа, между фигурных скобок.
}
\end{lstlisting}

\end{frame}

